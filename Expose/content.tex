\newcommand{\feat}{\texttt}

%\paragraph*{Abstract}
%\begin{abstract}
%	TODO
%\end{abstract}



%      \begin{figure}[!htb]
%        \center{\includegraphics[width=0.5\textwidth]
%        {images/20190419001057_1.jpg}}
%        \caption{\label{fig1} Player performance, accessable in-game: Number of normal attacks, number of CA, number of CA-Chips, etc.}
%      \end{figure}

%  (fig. \ref{fig1}, \ref{fig2}) 
%  An example is Chitnis et al.\cite{chitnis} 


\section{Motivation}
Object detection or object recognition is one of the most common problem across scientific fields. In the scope of this project we want to implement such a system. Under the pretext of entering the Kaggle competition, Aerial Cactus Identification, we want to study the fundamental practice of convolutional neural network (CNN). The goal of the competition is to assess the impact of climate change on flora and fauna. The sysytem VIGIA, an autonomous surveillance of protected areas, was build for the exact purpose. An important part of such as system is to be able to identify certain vegetation within the protected areas.




\section{Data}
The dataset consists of 17,500 aerial images out of which 13,136 contain an columnar cacti  and 4,364 do not.  Each image is 32x32 in size. The images have been resized by kaggle to be uniform in size. The dataset also contains an .csv file that annoates for each image if it contains an columnar cacti or not. The images are all from the Tehucan-Cuicatlan valley in the south of Mexico.  \cite{LOPEZJIMENZ2019} The images have been obtained using a drone from an flight altitude of 100 m. The images were manually labeled.
      \begin{figure}[!htb]
        \center{\includegraphics[width=0.5\textwidth]
        {images/test.jpg}}
        \caption{\label{fig1} test \cite{whit}}
      \end{figure}