\newcommand{\feat}{\texttt}

%\paragraph*{Abstract}
%\begin{abstract}
%	TODO
%\end{abstract}



%      \begin{figure}[!htb]
%        \center{\includegraphics[width=0.5\textwidth]
%        {images/20190419001057_1.jpg}}
%        \caption{\label{fig1} Player performance, accessable in-game: Number of normal attacks, number of CA, number of CA-Chips, etc.}
%      \end{figure}

%  (fig. \ref{fig1}, \ref{fig2}) 
%  An example is Chitnis et al.\cite{chitnis} 


\section{Motivation}
Object detection or object recognition is one of the most common problem across many different fields such as medical research or public safety. In the scope of this project we want to implement such a system. Under the pretext of entering the Kaggle competition, Aerial Cactus Identification, we want to study the fundamental practice of deep learning based approach. The goal of the competition is to assess the impact of climate change on flora and fauna. The system VIGIA, an autonomous surveillance of protected areas, was build for this exact purpose. An important part of such as system is to be able to identify certain vegetation within the protected areas. In the context of the competition we want to determine whether an image contains a columnar cactus of the neobuxbaumia tetetzo species or not. A past experiment with LeNet-5 showed an accuracy of 0.95.

