\documentclass[10pt,a4paper,twocolumn,DIV18]{scrartcl}   

\PassOptionsToPackage{hyphens}{url}\usepackage[linktocpage,bookmarks]{hyperref}
\hypersetup{    pdfauthor={Heinrich, Hsu, Schneider},
                pdftitle={DeepLearning Project Expose},
                pdfsubject={},
                colorlinks=true,
                citecolor=black,
                linkcolor=black,
                urlcolor=black}
\usepackage{graphicx}
\usepackage{amsmath}
\usepackage{url}
\usepackage[numbers]{natbib}
\usepackage{textcomp}

\pdfstringdefDisableCommands{
	\let\cite\@gobble % remove also first argument
}
\pdfstringdefDisableCommands{
	\let\\\empty % remove also first argument
}
\title{Expos\'e - Deep Learning}
\subtitle{Recognize artwork attributes using an Convolutional Neural Network}

\author{Matthias Heinrich, Wei-Hung Hsu, Tim Schneider}

\date{April 22th, 2019}

\RedeclareSectionCommand[%
	beforeskip=-0.5\baselineskip
]{paragraph}

\begin{document}  

\maketitle
\begin{abstract}
    The Metropolitan Museum of Art in New York has one of the worlds largest collection of historic artifacts.
    Annotating those with superficial visual attributes enables us to cluster them by visual appearance and thus search for visually related objects.
    Since creating those annotations by hand is time-consuming and error-prone, we automate this process by Convolutional Neural Networks.
\end{abstract}
\section{Motivation}
Object detection or object recognition is one of the most common problem across many different fields such as medical research or public safety. In the scope of this project we want to implement such a system. Under the pretext of entering the Kaggle competition, Aerial Cactus Identification, we want to study the fundamental practice of deep learning based approach. The goal of the competition is to assess the impact of climate change on flora and fauna. The system VIGIA, an autonomous surveillance of protected areas, was build for this exact purpose. An important part of such as system is to be able to identify certain vegetation within the protected areas. In the context of the competition we want to determine whether an image contains a columnar cactus of the neobuxbaumia tetetzo species or not. A past experiment with LeNet-5 showed an accuracy of 0.95.



\section{Data}
The dataset was obtained from kaggle\cite{imet}.
It contains 109,237 images of museum artifacts from the Metropolitan Museum of Art in New York.
The exact source of the images is unknown.
All objects have been labeled by  annotators without verification, thus noisy data is to be expected.
Each annotator was asked to label what he sees in the image.
Labels can describe what culture this object belongs too, or what the object was used for.
For example the image in figure \ref{fig:1} was annotated with: german, nymphenburg, children, female nudes and utilitarian objects.
Figure \ref{fig:2} shows some more example data.
\begin{figure}[h]
    \includegraphics[width=0.33\textwidth]{images/1}
    \caption{An teapot}
    \label{fig:1}
\end{figure}
Furthermore 7,443 images are available for verification.
Both the training and verification image-sets have an corresponding csv-file where one row corresponds with one image.
The images are labeled using numbers, an additional csv-file is given containing the corresponding attributes for each number.
There are 1103 labels in total.
\begin{figure}[h]
    \includegraphics[width=.2\textwidth]{images/3}
    \includegraphics[width=.2\textwidth]{images/4}
    \caption{Example images from the dataset}
    \label{fig:2}
\end{figure}

\section{Method}
In recent years, Convolutional Neural Networks have seen remarkable success in object classification~\cite{alexnet} and style detection~\cite{style}.
Similar to \citet{alexnet}, the number of different classes to choose from is huge (1103 distinct annotations).
However, unlike \citeauthor{alexnet}, where each image was classified as exactly one class, each artifact can have multiple annotations.
Nevertheless, we will likely use the a similar architecture as \citeauthor{alexnet}, since this enables us to use their pretrained convolutional layers.

\bibliographystyle{plainnat}
\bibliography{bibliography}


\end{document}