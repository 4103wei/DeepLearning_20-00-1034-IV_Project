\section{Motivation}
% labelling the objects will help experts search more effiiciently for visually related objects
% enables building a database of artifacts
The Metropolitan Museum of Art in New York, commonly referred to as ``The Met'', has a huge collection of over 1.5 million historic artifacts, comprised of different eras, cultures and styles.
All of these artifacts have been annotated by domain experts with attributes including, but not limited to, multiple object classifications, artist, title, period, date, medium, culture, size, provenance and geographic location.
While these annotations provide explanation on the history of these objects, none of them describes the object's visual appearance on a fine-grained superficial level.
Having such annotations would enable us to cluster visually related objects and thus enable people to efficiently search for objects by visual appearance.
Especially for non domain experts, these annotations can speed up the identification of an unknown artifact significantly, as they can search for similar objects in the museums database.
However, acquiring visual annotations for each of the 1.5 million artifacts is extremely time-consuming and error prone, since different people might have inconsistent approaches in labelling the objects.
Thus, the objective of this work is to create a system that automatically generates annotations from pictures of artifacts.


